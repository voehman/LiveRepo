\subsection{Aerodynamic \& Engine Forces}

The forces and moments section is devoted to the largest parts primaliry giving output in forces and moments. The top level subsystems \textsc{Engine} and \textsc{Aerodynamics} are described further on.

% rem to declare how notation in sub, super etc, bold sc it
% first dec of $\delta_{T}$, $\delta_{Npitch}$ $\delta_{Nyaw}$
% ok to have commas sometimes and paranthesis someitmes?
\subsubsection{Engine System}

The engine system is a subsystem accessible form the top level of the overall system, right next to the aerodynamics system, and is built to take the throttle command $\delta_{T}$, the nozzle pitch command $\delta_{Npitch}$, and the nozzle yaw command $\delta_{Nyaw}$ from the \textsc{Pilot Control} subsystem and to output forces and moments generated by the propulsion in effect. The thrust input is passed through a gain block increasing the power of the 0 to 1 signal with the maximum thrust, $F_{MAX}=160~N$. The thrust signal and the nozzle pitch and yaw signals each go through an individual second order non-linear actuator block, and into a "Mux". The signal contains all the information needed to project the nozzle forces in the body frame. The signal is routed to \textsc{Forces} where \textbf{$V_T$} is composed by projecting the force created on the vector components, $V^T_x$, $V^T_y$ and $V^T_y$, using the nozzle angles provided.

% EQ showing functions in like vector form I guess

% Add some talk about details like
% 1/0 throttle
% damping
% nozzle and angles, pictures for ease, equations too


% lots of delims i guess
% not sure about name for forces and how to split, is it good now?
% which are coefficients which are derivatives dCalpha/dt is deriv etc
% quick reminder since MÅLGRUPP is someone who knows some, MENTION THIS
% V (airspeed) is mentioned already
% does this bar thing work like this when rendered?
\subsubsection{Aerodynamics}

The \textsc{Aerodynamics} subsystem takes the data from the previous processes in \textsc{Environment}, \textsc{Actuators} and "Plant Data", the output from the Dynamics \& Visuals section, and calculates the forces and moments in the body frame. These forces and moments are summed up together with the propulsion forces and moments from the engine. The signals containing these forces and moments lead to the Dynamics \& Visuals section and the top level subsystem \textsc{6DoF Dynamics}. Inside \textsc{Aerodynamics} there are two subsystems, \textsc{Aerodynamic Stability} and \textsc{Control Surfaces}. These are separate subsystems as they deal with different sets of coefficients and derivatives but they both produce forces and moments in the body axis, which are summed up together in \textsc{Aerodynamics} and, where also the dynamic pressure $\barq$ is attained by

% dynamic pressure qbar equation

A quick reminder of theoretical aerodynamics used, and notations used in the simulation model, follows down below.

% use image/figure like one for axis systems
% different alphas? betas?
% use Cz and translate CD and CL into Cz etc for professionalism
% explain stability too, with the DCMs body to stab to wind and back etc
Figure \ref{BodyAlphaBetaV} shows how the wind vector, the airspeed, relates to the body frame by introducing $\alpha$ and $\beta$.


\label{BodyAlphaBetaV}









% What is missing from here?
% Delimitations?
