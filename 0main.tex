\documentclass[11pt,a4paper,twoside]{article}
\usepackage[top=70pt,bottom=80pt,left=78pt,right=76pt]{geometry}
\usepackage[utf8]{inputenc}
\usepackage[english]{babel}
% \selectlanguage{english}
\usepackage[T1]{fontenc}
\usepackage{lmodern}
% \usepackage{booktabs}
\usepackage{ctable}
\usepackage{graphicx}
\graphicspath{{./images/}}
\usepackage{subfig}
\usepackage{longtable}
\usepackage{amsmath, amssymb}
\usepackage{natbib, url}
% \usepackage[chapter, nottoc]{tocbibind} % Fix heading in bibs etc
\usepackage{array}
\usepackage{float}
\usepackage{caption}
% \usepackage{subcaption}
\usepackage{epstopdf}
% \usepackage{gensymb}
\usepackage{listings}
\newcommand*\PBS[1]{\let\temp=\\#1\let\\=\temp}

\begin{document}

\newlength{\figurewidth}\setlength{\figurewidth}{0.95\textwidth}
\begin{titlepage}
  \begin{center}

% 	\includegraphics[width=4cm]{LogoEng}
    % \vspace{1.5cm}

    \begin{LARGE}
      \bfseries
      CFD Study of 2D Airfoil and 3D Wing To Improve Relations of Lift and Drag\\
    \end{LARGE}

    \vspace{1cm}

    \begin{Large}
      \bfseries
       Victor Öhman\\[1ex]
    \end{Large}

    \vspace{1cm}
    \vspace{1.5cm}
    \begin{small}
      Division of Fluid Dynamics\\
			Department of Management and Engineering\\
      Linköpings Universitetet\\
      SE-581 83 Linköping, Sweden\\

    \end{small}
    \vspace{1cm}
    % \rule{\textwidth}{0.1mm}
     \vspace{0.5cm}
		 \begin{small}
      \bfseries
    \end{small}
	\end{center}
\end{titlepage}
\raggedbottom
\thispagestyle{empty}
\cleardoublepage

%\frontmatter


\chapter*{Abstract} %informative (not descriptive)
\clearpage
\thispagestyle{empty}
\cleardoublepage
\thispagestyle{plain}


\chapter*{Acknowledgments}
\clearpage
\thispagestyle{empty}
\cleardoublepage
\thispagestyle{plain}


\chapter{Notational Conventions}
\label{cha:notation}
\vspace{-2ex}


\section*{Abbreviations and Acronyms}
\vspace*{-2ex}
\begin{longtable}{p{.25\textwidth}p{.65\textwidth}} % Abbreviations

 \multicolumn{1}{l}{\bfseries Abbreviation} &
 \multicolumn{1}{l}{\bfseries Meaning}\\

\endhead
\endfoot

LiU     & Linköping University\\
LiTH	  & Linköpings tekniska högskola \\
DoF		  & Degrees of Freedom \\
CAD     & Computer aided design\\
PWM     & Pulse-width Modulation\\

\end{longtable}
\vspace*{1ex}


\section*{Symbols and Mathematical Notation}
\vspace*{-2ex}
\setlength\extrarowheight{1pt}
\begin{longtable}{>{$\displaystyle}p{.25\textwidth}<{$}p{.65\textwidth}} % Math
  \multicolumn{1}{l}{\bfseries Notation} &
  \multicolumn{1}{l}{\bfseries Meaning}\\
\endhead
\endfoot
a 												& Small letter\\
C_{D} 										& Large letter\\
\vec n 										& Normal vector\\
p 												& Pressure\\
t 												& Time\\
\alpha 										& Angle\\
\end{longtable}


\clearpage
\thispagestyle{empty}
\cleardoublepage
\tableofcontents

\clearpage
\thispagestyle{empty}

\listoffigures

\clearpage
\thispagestyle{empty}

\listoftables

\clearpage
\thispagestyle{empty}

%\mainmatter

\section{Introduction \& Overview}

\subsection{Overview}

A sub-scale model of an aircraft can be used to test out various scenarios during flight that would be too expensive and dangerous to test out with the real aircraft. It is always a good idea to make a sub-scale model before building the real full scale aircraft. Although a well made sub-scale model is cheaper than a full scale aircraft it is still expensive to build anew if crashed, thus much thought and time is allocated to these models so they are as good as possible at the time they are built. Simulation models in computers can and should be utilized in parallel with the actual flight testing so that a reliable computerized model of the aircraft can be used instead of the sub-scale aircraft while testing new or advanced maneuvers. A goal for acceptable inaccuracy between the recorded flight data and the results from the computer simulation model has to be set by the designer.

\input{1background}
\input{1taskandstrategy}


\newpage
\input{2AircraftModel}

\newpage
\section{Aircraft Simulink Model}

\subsection{Preface}

From this part in the report existing subsystems in the model are addressed with a bold typeface.  
 % short
\subsection{Blocks \& Definitions}

This part contains a summary of some of the commonly used blocks in the model along with the function and some specific properties not explained elsewhere.

% What is missing from here?
% Delimitations?
 % maybe common + pictured
\subsection{General Model Outline}

A visual representation of the model is shown below in figure \ref{fig:schematic-modeloutline}. This interpretation of the system contains the most important areas and how they depend on eachother for information. This will make it easier to inspect these areas in detail as we go along.

%%%%%%%%%%%%%%%%%%%%%%%%%%%%%%%%%%%%%%%%%%%%
%                                          %
%                                          %
%                                          %
%                                          %
%                                          %
%                                          %
%       fig:schematic-modeloutline         %
%                                          % Shade code along control
%                                          % and flight data
%                                          %
%                                          %
%                                          %
%                                          %
%                                          %
%%%%%%%%%%%%%%%%%%%%%%%%%%%%%%%%%%%%%%%%%%%%

Since the model should be able to make use of live input from a pilot as well as actual flight data from a timeseries, the schematic is shaded along the logical path of using recorded flight data. The parts of certain interest that are explained further into the report are represented in the list as follows.

\begin{itemize}
  \item Pilot Control
    \begin{itemize}
      \item Flight Data
      \item Live Flight
    \end{itemize}
  \item Engine
  \item Aerodynamics
  \begin{itemize}
    \item Flight Data
    \item Live Flight
  \end{itemize}
  \item Dynamics
  \begin{itemize}
    \item 3DoF for Longitudinal Stability
    \item 6DoF Complete Model
  \end{itemize}
  \item Visualization
  \item Environment
\end{itemize}

The part using 3DoF while being important is in essense a tool used for the purpose of investigating longitudinal stability before connecting the 6DoF to also assess lateral stability. It is thus a included in the part explaining the 6DoF model and does not need to be explained separately. 
 % showing nice looking schematic
\subsection{Pilot Control	 \& Actuators}

\subsubsection{Pilot \& Control}

The part of the system named \textbf{Control} is the part of the model where the pilot commands are either created or read, depending on if the model is run with live pilot control or is run with recorded data. The pilot control PWM signals (Pulse-Width Modulation) are the command signals representing specific SI values. Values of which they are translated into immediately in the same model block before they are directed into the \textbf{Actuators} subsystem.

\subsubsection{Actuators \& Servos}

The part of the system named \textbf{Actuators} is the part of the model designed to take values from the \textbf{Pilot \& Control} subsystem and run them through the appropriate  output these commands in thrust and control surface deflection angles deflection


% What is missing from here?
% Delimitations?
\subsubsection{Sensors}
 % control and actuator focus
\input{3forcesmoments} % engine prop, weight, aero
\subsection{Aircraft Flight Dynamics}

The fourth and last major part in the top level of the model is the aircraft dynamics part. The subsystem called \textbf{Dynamics} takes the input in terms of forces and moments and outputs the so called \textit{Plant Data} which is then used throughout the model and could be looked at as the current state of the model for the coming timestep. The data collected in \textit{Plant Data} is explained later.

% Explain plant data
% What is missing from here?
% Delimitations?

\subsection{Summary}

% TEST

\begin{table}[H]
\caption{Test table for list of tables.}
\centering
\begin{tabular}{lccr}
 & \textbf{Coarse}& \textbf{Fine} & \textbf{Relative difference} \\
Mesh nodes & 251486 & 569727 & 127\% \\
$C_D$ & 0.05896 & 0.057497 & -2.5\% \\
$C_L$ & 1.3541 & 1.3796 & 1.9\% \\
\end{tabular}
\label{meshilainen3dII}
\end{table}

\begin{figure}[H]
 \centering
 \includegraphics[width=10cm]{TESTimg.png}
 \caption{Test figure for list of figures.}
\label{Cd_mesh_ind}
\end{figure}
 %





\newpage
\bibliographystyle{unsrt}
\nocite{*}

\bibliography{ref}

\end{document}

% \begin{equation}
% C_{d,\,induced} = \frac{C_{l}^{2}}{\pi\, AR\, e}
% \label{DragEq}
% \end{equation}

% \begin{table}[H]
% \caption{Spatial discretization schemes and the pressure-velocity coupling for the 2D airfoil case.}
% \vspace{0.2cm}
% \centering
% \begin{tabular}{ll}
% Pressure-Velocity Coupling & Coupled \\
% Spatial Discretization $\rightarrow$ \textbf{Gradient} & Least Square Cell Based \\
% Spatial Discretization $\rightarrow$ \textbf{Pressure} & Second Order \\
% Spatial Discretization $\rightarrow$ \textbf{Density} & Third Order MUSCL \\
% Spatial Discretization $\rightarrow$ \textbf{Momentum} & Second Order Upwind \\
% Spatial Discretization $\rightarrow$ \textbf{Intermittency} & Second Order Upwind \\
% Spatial Discretization $\rightarrow$ \textbf{Turbulent Kinetic Energy} & Third Order MUSCL \\
% Spatial Discretization $\rightarrow$ \textbf{Specific Dissipation Rate} & Third Order MUSCL \\
% Spatial Discretization $\rightarrow$ \textbf{Energy} & Second Order Upwind \\
% Pseudo Transient & On \\
% High Order Term Relaxation & On \\
%  & Second Order Implicit
% \end{tabular}
% \label{solvmeth}
% \end{table}

% \begin{table}[H]
% \centering
% \caption{Solution controls used for in the Ansys solver for the 2D airfoil.}
% \begin{tabular}{lr}
% Pressure & 0.3 \\
% Momentum & 0.2 \\
% Turbulent Kinetic Energy & 0.6 \\
% Specific Dissipation Rate & 0.5 \\
% Energy & 0.9 \\
% \end{tabular}
% \label{solcont}
% \end{table}

% \begin{equation}
% \label{bl_flat_plate}
% \delta_{max} = max(0.382xRe_{x}^{-1/5})
% \end{equation}

% \begin{table}[H]
% \centering
% \caption{The settings and quality of the coarse and fine mesh.}
% \begin{tabular}{lccc}
% \multicolumn{1}{c}{\textbf{Mesh settings}} && \textbf{Coarse} & \textbf{Fine}\\
% Number of inflation layers && 51 & 51\\
% Growth rate of inflation layers && 1.1 & 1.1\\
% First layer thickness &$[m]$ & 6e-6  & 5e-6 \\
% Number of divisions && 1101 & 1351\\
% Body of influence element size &$[m]$ & 6e-2 & 6e-2 \\
% Bulk max size/max face size &$[m]$ & 0.20/0.20  & 0.10/0.10 \\
% Bulk growth rate &&1.050 & 1.050\\
% Total number of nodes && 180000 & 245000\\
% \multicolumn{1}{c}{\textbf{Mesh quality}} && \textbf{Coarse} & \textbf{Fine}\\
% Max aspect ratio && 179.13 & 193.81\\
% Average aspect ratio && 15.694 & 15.418\\
% Max skewness && 0.72632 & 0.7631\\
% Average skewness && 0.13813& 0.12419\\
% \end{tabular}
% \label{meshsettings}
% \end{table}

% \begin{table}[H]
% \caption[Mesh independence 2D.]{The results of the two different meshes combined with the relative difference between them.}
% \centering
% \begin{tabular}{lccr}
%  & \textbf{Coarse}& \textbf{Fine} & \textbf{Relative difference} \\
% Mesh nodes & 180 000 & 245 000 & 36.1\% \\
%  $C_D$ & 0.00790 & 0.00740 & 6.73\% \\
% $C_L$ & 0.368 & 0.366& 0.450\% \\
% $\frac{L}{D}$ & 46.3 & 49.42 & 6.74\% \\
% \end{tabular}
% \label{meshilainen}
% \end{table}

% \begin{table}[H]
% \centering
% \caption{Lengths and diameters in the computational domain for the 3D wing case I.}
% \begin{tabular}{lcc}
% Domain Forward/Backward 	&$[m]$ 		& 6.0/20 \\
% Domain Left/Right 			&$[m]$ 		& 8.0/8.0 \\
% Domain Up/Down 				&$[m]$ 		& 5.0/5.0 \\
% \end{tabular}
% \label{domainsizeetc3d1}
% \end{table}

% \begin{table}[H]
% \centering
% \caption[Computational domain and bluff body geometry, 3D wing case I.]{Measurements in the computational domain for the 3D wing case I along with the bluff body measurements. The origin is located in the wing root at leading edge, forward is positive x-axis and left is as wing extends.}
% \begin{tabular}{lcc}
% Bluff Body Forward/Backward 	&$[m]$ 		& 1.0/3.0 \\
% Bluff Body Straight Up/Down		&$[m]$ 		& 1.4/1.6 \\
% Bluff Body Radius From Front	&$[m]$ 		& 0.3 \\
% Bluff Body Radius Along Sides	&$[m]$ 		& 0.2 \\
% BoI I Forward/Backward 			&$[m]$ 		& 2.0/4.0 \\
% BoI I Left/Right 			 	&$[m]$ 		& 4.2/2.0 \\
% BoI I Body Up/Down 			 	&$[m]$ 		& 1.4/1.4 \\
% BoI II Backward From BoI I 		&$[m]$ 		& 16 \\
% BoI II Left/Right 			 	&$[m]$ 		& 4.2/2.0 \\
% BoI II Body Up/Down 		 	&$[m]$ 		& 0.8/0.8 \\
% \end{tabular}
% \label{domainsizeetc3d2}
% \end{table}

% \begin{table}[H]
% \centering
% \caption{The settings and quality of the mesh.}
% \begin{tabular}{lcc}
% \multicolumn{3}{c}{\textbf{Mesh settings}}  \\
% Bulk max size/max face size &$[m]$ & 0.80/0.80  \\
% Min size &$[m]$ & 0.10  \\
% Bulk growth rate && 1.30 \\
% Total number of nodes && 222000 \\
% \multicolumn{3}{c}{\textbf{Mesh quality}} \\
% Max skewness && 0.799 \\
% Average skewness && 0.220
% \end{tabular}
% \label{meshsettings13d}
% \end{table}

% \begin{table}[H]
% \centering
% \caption{Body sizing for the BoI's and edge sizing along the leading edge.}
% \begin{tabular}{lcc}
% \multicolumn{3}{c}{\textbf{Mesh quality}} \\
% BoI I element size 		&$[m]$ & 0.13 \\
% BoI II element size 	&$[m]$ & 0.20 \\
% BoI I min size 			&$[m]$ & 0.08 \\
% BoI II min size 		&$[m]$ & 0.08 \\
% Edge sizing along leading edge &$[m]$& 0.10 \\
% \end{tabular}
% \label{boiboiboi3d}
% \end{table}

% %Use Bibtex for references and the \texttt{cite} command to refer to the sources. See the .bbl file. Note - there are more available entries than you find in the file. Add your references as appropriate entries and compile at least three times. Citation: The first is to a web page~\cite{website:Tornado}. The second to a journal paper (article)~\cite{laurence:boundary}, and the third to a book~\cite{blazek:cfd}.

% %\begin{equation}
% %\alpha = \sqrt{\alpha^{2}}
% %\label{root}
% %\end{equation}
% %\pagebreak
% %\chapter{Bibliography}


% \pagebreak
% \appendix
% \chapter{Matlab Code for NACA 4-digit series airfoil}
% \lstset{columns=fullflexible, basicstyle=\ttfamily}
% Please note that if the resolution $n$ is set too high there might be problems when importing the coordinate file into Workbench Geometry.
% \begin{lstlisting}
% % function[M1,M2] = NACA4digit(NACA,c,n,cut) %Comment NACA,c,n and cut in
% % code if function calling is to be used

% clear all
% close all
% clc


% \end{lstlisting}
