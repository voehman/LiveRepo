% v1
% delims further down
% controller 2 l?
\subsection{Pilot}

The pilot input in this project can either be by direct input from a controller or by logged data from an actual flight test with the demonstrator. This project only made use of the logged data option. The top level subsystem \textbf{Pilot Control} has a manual switch allowing the user to choose between logged data or direct control via controller.

% origin of equations to convert PWM
% introduced \delta_{ca} \delta_{els} \delta_{elp}\delta_{ru} \delta_{T} \delta_{lg}
% need prev talk about base workspace, maybe overall generla somt
% delim/unmade/recom - trim in here? finding out trim or static stuff with the static one
% thrust vectoring notation or whatever its called
% talk about LG not used
\subsubsection{Logged Data}

Leaving the switch to accept data from prerecorded data from pilot control of a flight test with the demonstrator means that all information is taken from a timeseries object. The pilot control data used in this project was saved in a file called log\_pilotcmd.mat which is a timeseries object containing 11 values recorded for each timestep during the flight test. The signals are recorded from the controler and are \textit{PWM, Pulse Width Modular} signals.

The signals are taken in by importing the file with the logged data from the base workspace, where it was initially imported as a variable along with the rest of the initial conditions and declared variables and constants. The signals are split two ways, one of which leads to a subsystem that makes use of the signals where the signals are split, using "Demux", and then converted one by one by function blocks according to equations by alejandro???????, giving the deflection angles in degrees ($\delta_{ca}$ for canard, $\delta_{els}$ and $\delta_{elp}$ for elevons starboard and port and $\delta_{ru}$ for rudder), thrust ($\delta_{T}$) from 0 to 1, thrust vectoring in degrees ?????, landing gear ($\delta_{lg}$) as 0 or 1 and clean pilot controler signals for roll, pitch and yaw. Landing gear effects on simulation are zero since this is not included in the model, although the signal is included in the bus with the signals for the deflection angles (excluding thrust vectoring angles). The other subsystem that the signals lead to is one built mainly for testing purposes, where the deflection angles are flat at zero degrees after conversion, the thrust is static and controled by a constant (by choosing a value from 0 to 1). The remaining signals are left untouched, since they do not alter the flight.

The signals with regard to thrust are separated from those controling deflection angles and landing gear. The subsystem \textbf{Pilot Control} contains manual switches giving the user the option to use the real signals or to use testing signals of each thrust and deflection angles on control surfaces.

If a logged data case is used and aerodynamic coeficient data is available, that data can be gathered in the same way as the pilot control values (from a timeseries object existing in the base workspace), within the \textbf{Control Surfaces} subsystem, located within the top level subsystem \textbf{Aerodynamics}.

% Delim not used, maybe write here and in delim to fill out and fo rease since mayube annoying to read delim first and remember it?
\subsubsection{Live Piloting}

The live piloting model available with a manual switch in \textbf{Pilot Control} is as of yet not implemented. This is due to time a time constriction and to prioritize the validation of the model against actual recorded fligth data.
