% UNDONE
\subsection{Aircraft Flight Dynamics}

The subsystem called \textsc{Dynamics} takes the input in terms of forces and moments and outputs the so called \textit{Plant Data} which is then used throughout the model and could be looked at as the current state of the model for the coming timestep. The data collected in \textit{Plant Data} is explained later.

% v1 v2
% Missing LLA (how ned it?)
% Eq undone
% abbrev NED
% missing image
% 1 delim
% use word cartesian somehwere, maybe euler too?
\subsubsection{Inertial Reference Frame}

In simple terms, two axis systems are needed to build an aircraft model (and the majority of models with moving parts as well for that matter), one that is fixed and decides where the origin is located and one that is attached to the moving body. In this project the fixed reference frame, \textbf{$X_e$}, is a vector composed of the position in the x direction $X^e$, the position in the y-direction $Y^e$ and the altitude of the aircraft in the z-direction $Z^e$. The practical use for these values in the model is to declare them as the initial values for each and any simulation, hence, these values are found as follows.

\begin{equation}
 nonumber Xe => Xe_0
 nonumber Ye => Ye_0
 nonumber Z_e => $h\_{ini}$
\end{equation}

%%%%%%%%%%%%%% IMAGE OF FIXED AND BODY AND LINE BETWEEN

%
% Delimitation
%
To make use of Newton's laws an intertial frame has to be defined. In this project the earth's surface is assumed flat and non-rotating.
%

To know which way the frame should be positioned a \textsc{NED, North, East, Down} system is used, making the positive x-axis directed North, the positive y-axis directed East and the positive z-axis directed down towards the surface of the earth with the origin in the moving body. These are the values that are used as the output position in every timestep of the simulation.

% v1
% abbrev CG, Vb, AC (undeclared in text), \alpha
% stab?
% use word cartesian somehwere, maybe euler too?
\subsubsection{Body \& Wind Frame}

The AC body frame is an axis system with the origin located at the CG of the body, the positive x-axis parallel to fuselage forward axis in the aircraft body, the positive y-axis on the right hand side of the pilot (starboard) and the positive z-axis directed down towards the ground. A vector, \textbf{$V_b$}, consisting of $V^b_x$, $V^b_y$ and $V^b_z$ constitutes the body velocity in these directions and can be seen as a vector of projections of the relative wind on the body axes. The rotations along each axis are the Euler rotation angles $\phi$ for roll around the x-axis, $\theta$ for pitch around the y-axis and $\psi$ for heading rotation around the z-axis. Figure \ref{BodyWindFrameRot}

% IMAGE
\label{BodyWindFrameRot}

The body frame is rotated with a \textsc{DCM, Directional Cosine Matrix}
% this needs to be filled out, this is with regard to euler angles, alpha and beta can be described more in aerodynamics
The wind frame, with vector named \textbf{$V_w$} is a lot like the body frame, but has the x-axis directed along the actual flight path. In practice this is the calculated airspeed of the aircraft as the first component of the vector, with the projections of the second and third component at zero. The airspeed from this point will be called V.

% show V_w in a vector with the projections (if I didnt brainfart here)

As output values the body velocity \textbf{$V_b$} represents the translational characteristics over the timesteps with

\subsubsection{Equations of Motion}

\subsubsection{Longitudinal Stability}
\subsubsection{Lateral Stability}



% Explain plant data
% What is missing from here?
% Delimitations?
